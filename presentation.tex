\documentclass[11pt, a4paper]{article}

\usepackage[utf8]{inputenc}
\usepackage[T1]{fontenc}
\usepackage[francais]{babel}
%\usepackage{xcolor}
\usepackage{amsmath}
\usepackage{amssymb}
% Rappel%Pour faire un symbole comme R pour les réels:
% $\mathbb{R} $

\begin{document}
%Définition de commandes
%\newcommand{\impo}[1]{\emph{#1}}
%Fin de définition de commandes

\begin{abstract}
	L'objectif de ce projet est de calculer une solution de l'équation de Laplace à l'aide de la
	méthode de Monte Carlo.
\end{abstract}

\section{Présentation de la méthode de Monte Carlo}

La méthode de Monte Carlo s'appuie sur les propriétés de la marche aléatoire. 

\section{Schéma retenu}

\subsection{Discrétisation}
On commence par discretiser le domaine choisi. On a choisi de quadriller le domaine.  Au début, nous
avons commencé par utiliser les coordonées des points du plan puis executer la marche aléatoire en
«sautant» de points en points où la distance entre chaque saut est donnée par $1/N$, où $N$ était un
paramètre du schéma.  Malheuresement, au fur et à mesure des pas, à cause des erreurs d'arrondi, on
«sort» de la discrétiasation du domaine. Nous avons donc préféré construire une grille (une matrice)
qui réprensente les points du domaine, puis effectuer la marche aléatoire dans cette grille. Les
indices d'une matrice étant entier, il n'y a plus d'erreur d'arrondi possible.

La discrétisation d'effectue donc grâce à une matrice remplie de $1$ ou $0$ pour indiquer
respectivement si un point se trouve ou non dans le domaine.

\subsection{Marche aléatoire}

Il faut maintenant parcourir tout les points de la grille qui sont à l'intérieure du domaine.
Pour chaque point du domaine, on lance $K$  marches aléatoires qui commencent en ce point. On fait
ensuite la moyenne de toutes les valeurs données par la fonction au bord en tout les points à
l'extérieure du domaine atteint après une marche aléatoire.


\subsection{Paramètres du shéma}

\begin{itemize}
	\item $N$ correspond à la taille de la grille qui discrétise le domaine. Plus ce nombre est
		élevé, plus la discrétisation est fine.
	\item $K$ nombre de marche aléatoire effectuée pour chaques points du plan.
\end{itemize}

\section{Convergence}

Pour analyser la convergence de ce schéma aléatoire nous avons du ??

\section{Avantage de la méthode de Monte carlo}

On voit tout de suite les intêret de cette méthode pour le calcul de la solution de l'équation de
Laplace:
\begin{itemize}
	\item La relative simplicité du schéma: la seule difficulté qui s'est présentée fut de créer
		une grille pour lancer la marche aléatoire.
	\item Une parallélisation du calcul possible: la solution est construite point par point
		(!). En effet, contrairement aux schémas classiques qui calculent la valeur en
		chaque point en faisant la moyenne des points autour, ce schéma aléatoire peut par
		exemple calculer la solution en un point précis. On remarque aussi que l'on peut
		facilement stopper le calcul d'une solution pour le reprendre plus tard: il suffit
		de faire une moyenne des deux résultats.
\end{itemize}
Qui s'accompagne évidemment de désavantages, qui sont néanmoins gérables:
\begin{itemize}
	\item Le bon fonctionnement de l'algorithme dépend en pratique d'un bon générateur de
		hasard. Ce sujet dépasse (de loin) le cadre de ce projet, mais il serait bon de
		vérifier la qualité du hasard proposée par \emph{Scilab}

	\item La difficulté théorique pour justifier la convergence du schéma: il nous faut en effet
		les résultats issue de la théorie de la probabilité, et en particulier de la marche
		aléatoire.
	\item La convergence qui ne se fait que en $O(\sqrt{N})$. De plus, il est difficile de voir
		comment améliorer cette convergence. Contrairement aux schémas classiques où il
		suffit généralement d'un développement de Taylor à l'odre supérieure qui donne un
		schéma qui converge plus vite (au prix de calcul plus compliqué certe).
\end{itemize}

\end{document}
